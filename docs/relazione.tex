\documentclass[12pt, letterpaper]{article}
\usepackage{graphicx} %LaTeX package to import graphics
\usepackage[obeyspaces]{url}
\usepackage{hyperref}
\usepackage{subcaption}
\usepackage{listings}

\graphicspath{{images/}} %configuring the graphicx package

\hypersetup{
      colorlinks=true,
      linkcolor=blue,
      filecolor=magenta,      
      urlcolor=cyan,
      pdftitle={Sviluppo di un contatore elettrico intelligente},
      pdfpagemode=FullScreen,
}


\title{
    \includegraphics[scale=0.5]{unibalogo.jpg}~\\[1cm]
    \textbf{Sviluppo di un contatore elettrico intelligente}
}
\author{Stefano Antonio Labianca}

\date{22 Gennaio 2024}
\renewcommand*\contentsname{Tabella dei contenuti}
\renewcommand{\figurename}{Figura}

\begin{document}

\maketitle


\textbf{matricola: } 758364
\hfill
\textbf{email: } s.labianca10@studenti.uniba.it

\par\noindent\rule{\textwidth}{0.4pt}~\\[5cm]

\tableofcontents ~\\[5cm]

\section{Introduzione}

Nell'arco della nostra giornata, usiamo diversi dispositivi elettronici e,
alle volte, anche per diverse ore della giornata o addirittura per tutto
il giorno. \\ \break
Per chi abita nelle zone di campagna, o in abitazioni singole, usare molti
dispositivi elettronici contemporaneamente, specialmente se hanno alti consumi o
possiedono una classe energetica bassa, fa scattare il salvavita.

\subsection{Dispositivo salvavita}

Il "salvavita", o più propriamente detto interruttore differenziale, è un dispositivo
che arresta il flusso di energia elettrica dal contatore di
un'abitazione, proteggendo persone e animali. \\ \break
Questi interruttori, monitorano la differenza di corrente in entrata
e in uscita dal dispositivo e, quando la differenza di corrente in entrata e in
uscita supera una certa soglia, allora l'interruttore scatta togliendo l'alimentazione
al circuito.

\begin{figure}
      \centering
      \includegraphics[scale=0.2]{interruttore-diff.jpg}
      \caption{Esempio di contatore differenziale}
\end{figure}


\subsection{Obiettivo del progetto}

Il progetto si pone l'obiettivo di sviluppare un programma in grado di svolgere
i seguenti task:

\begin{enumerate}
      \item Determinare da una lista di dispositivi, quali possono tenere accesi contemporaneamente
            senza che salti il salvavita.
      \item Tenere traccia dei dispositivi elettronici e del loro consumo in Watt.
      \item Ottenere tutti quei dispositivi che rispettano certi vincoli di consumo energetico.
\end{enumerate}

\section{Progetto}

\subsection{Inizializzare il progetto}

\subsubsection{Scaricare da GitHub}

Il primo passaggio è quello di clonare la repository cliccando al seguente
\href{https://github.com/Stefano-Labianca/smart-energy-controller}{link}. \\

\subsubsection{Impostare l'ambiente virtuale}

\noindent Va creato successivamente un ambiente virtuale. In questo modo l'interprete Python,
le librerie e gli script installati al suo interno, saranno isolati dagli altri ambienti
virtuali e da qualsiasi libreria installata sul proprio sistema. \\

\noindent Per creare l'ambiente virtuale, entrate nella cartella del progetto e
digitate il seguente comando:

\begin{verbatim}
      python -m venv .venv
\end{verbatim}

% Mettere citazione nella bibbliografia

\noindent Grazie a questo comando, verrà creata una cartella \path{/.venv} che conterrà tutto il
necessario per lavorare con l'ambiente virtuale. \\

\noindent Una volta creato, è necessario attivare l'ambiente virtuale.
Se siente in ambiente MacOS o Linux, digitate il comando

\begin{verbatim}
      source .venv/bin/activate
\end{verbatim}

\noindent Il file \path{activate} serve per "accendere" l'ambiente virtuale. \\

\noindent Se invece siete in ambiente Windows, allora posizionatevi prima dentro la cartella
\path{./venv/Scripts/} per poi  digitare uno dei seguenti comandi, in base al tipo di terminale in uso:

\begin{verbatim}
      activate.bat // Se usi il CMD
      .\Activate.ps1 // Se usi la PowerShell
\end{verbatim}

\noindent Se invece si sta usando il GitBash, in ambiente Windows, allora il comando da appliace è il seguente:

\begin{verbatim}
      source .venv/Scripts/activate
\end{verbatim}

\noindent L'ambiente virtuale sarà attivato con successo quando sul vostro terminale avrete
qualcosa di simile alla figura 2. \\

\noindent In caso di problemi nell'uso della PowerShell, è possibile andare alla
sezione: \hyperref[sec:powershell-error]{Risolvere il problema di esecuzione con la PowerShell}. \\


\begin{figure}[h]
      \centering

      \begin{subfigure}[h]{1\textwidth}
            \includegraphics{terminale-bash.png}
            \caption{Uso di GitBash}
      \end{subfigure}

      \vspace{0.5cm}

      \begin{subfigure}[h]{1\textwidth}
            \includegraphics{terminale-cmd.png}
            \caption{Uso del CMD}
      \end{subfigure}

      \vspace{0.5cm}

      \begin{subfigure}[h]{1\textwidth}
            \includegraphics{terminale-powershell.png}
            \caption{Uso della PowerShell}
      \end{subfigure}

      \caption{
            Risultato dell'attivazione dell'ambiente virtuale. In figura (a) abbiamo
            l'uso del GitBash, in figura (b) del CMD mentre infine, nella figura
            (c), abbiamo l'uso della PowerShell.
      }
\end{figure}


\subsubsection{Avviare il progetto}

Per avviare il progetto, bisogna tornare alla directory principale del progetto e installare
le dipendenze con il comando: \\

\begin{verbatim}
      pip install -r requirements.txt
\end{verbatim}

\noindent Una volta installate, digitare il seguente comando per avviare il programma

\begin{verbatim}
      python main.py
\end{verbatim}

\noindent In caso di errore, vedere la sezione:
\hyperref[sec:python-error]{Risolvere il problema di esecuzione con la PowerShell}.


\subsection{Struttura del progetto}

All'intero del progetto, possiamo trovare le seguenti cartelle:
\begin{itemize}
      \item \path{/.vevn}: Cartella contenente tutto il necessario per lavorare con
            l'ambientevirtuale.
      \item \path{/appliance}: Qui è possibile trovare la classe Appliance, che definisce
            un elettrodomestico, insieme ad una serie di metodi di suppporto,
            contenuti in \path{appliances_controller.py}

      \item \path{/cli}: Troviamo una classe che incapsula tutta la logica legata agli input
            e all'output del terminale

      \item \path{/csp_problem}: Questa cartella contiene tutti i file legati all'argomento del
            CSP. \\
            Infatti è possibile trovare la rappresentazione delle variabili e dei vincoli, fatta
            rispettivamente usando le classi Variable e Constraint. \\
            Inoltre è presente anche la classe CSP usata come wrapper per rappresentare un generico
            problema di questa categoria. \\
            Infine è presenta la cartella \path{/algorithm} che contiene le realizzazioni degli
            algoritmi DFS e GAC usati per risolvere i problemi legati al CSP.

      \item \path{/knowledge_base}: Contiene una classe usata per rappresentare il Sistema Esperto
            realizzato.

      \item \path{/ontology}: Questa cartella contiene una classe che permette di manipolare
            l'ontologia contenuta all'intero del file \path{appliance_ontology.rdf}.

      \item \path{/test}: Contiene tutti quei file contenente vari test fatti al programma.

      \item \path{/utils}: Contiene file di utilità che facilitano alcune operazioni interne al programma.
            Per esempio, il file \path{pagination.py} viene usato per impaginare l'output del programma.
\end{itemize}




% \section{CSP}

% \section{Ontologie}


% \section{Sistema esperto}

\section{Troubleshooting}

\subsection{Risolvere il problema di esecuzione con la PowerShell}
\label{sec:powershell-error}

In caso non si riesca ad eseguire con la PowerShell l'attivazione dell'ambiente
virtuale, provate a svolgere i seguenti passi. \\

\noindent Aprire innanzitutto la PowerShell come amministratore nella cartella del progetto.
Una volta aperta la PowerShell, digitare il comando:

\begin{verbatim}
      Get-ExecutionPolicy
\end{verbatim}

Grazie a questo comando, possiamo sapere quale execution policy è impostata per la PowerShell.
In figura 3, è mostrato una possibile execution policy impostata nella PowerShell. \\

\begin{figure}[h]
      \centering
      \includegraphics{powershell-error.png}
      \caption{Possibile execution policy}
\end{figure}

\noindent Per permettere l'esecuzione degli script \texttt{.ps1}, allora bisogna impostare
su \texttt{AllSigned} la execution policy. \\
\noindent Per farlo si usa il comando: \\

\begin{verbatim}
      Set-ExecutionPolicy -ExecutionPolicy AllSigned
\end{verbatim}

Una volta eseguito questo comando, andare nella cartella \path{/.venv/Scripts/} e digitare il
comando:

\begin{verbatim}
      .\Activate.ps1
\end{verbatim}

Apparirà sul terminale il seguente output: \\

\begin{figure}[h]
      \centering
      \includegraphics[scale=0.6]{terminal-message.png}
      \caption{Messaggio di conferma}
\end{figure}

Per eseguire lo script di attivazione, allora inserite \texttt{V} e
premete invio. \\ \break

\noindent Una volta che avete finito l'esecuzione del programma, potete anche reimpostare
la execution policy al suo stato orinario, usando il comando:

\begin{verbatim}
      Set-ExecutionPolicy -ExecutionPolicy <PolicyNamePrecedente>
\end{verbatim}

\subsection{Errore di esecuzione del programma Python}
\label{sec:python-error}

E' possibile che, durante l'esecuzione del programma, possa apparire il seguente
messaggio di errore: \\ \pagebreak

\begin{figure}[h]
      \centering
      \includegraphics[scale=0.55]{errore-python.png}
      \caption{Errore di esecuzione}
\end{figure}

\noindent Questo errore è dovuto ad una versione datata della libreria \texttt{frozendict}
usata come dipendenza della libreria \texttt{experta}. \\

\noindent Fare l'upgrade della libreria \texttt{frozendict} alla versione più recente, andrebbe a
creare dei conflitti di dipendenza tra le due versioni della librerie. \\

\noindent Per risolvere questo problema, bisogna andare nella cartella
\path{/.venv/Lib/site-packages/frozendict} e aprire il file \path{__init__.py}.

\noindent Una volta aperto, bisogna cambiare la seguente linea di codice: \\

\begin{lstlisting}[language=Python]
      class frozendict(collections.Mapping):
            ...
\end{lstlisting}

Nella seguente:

\begin{lstlisting}[language=Python]
      class frozendict(collections.abc.Mapping):
            ...
\end{lstlisting}

% \section{Conclusioni e Sviluppi Futuri}

% \section{Riferimenti Bibliografici}

\end{document}